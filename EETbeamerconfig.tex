
% DECLARACION DAS FONTES DA UVIGO
\usepackage[sfdefault]{roboto}
\usepackage{librebaskerville}
\setbeamerfont{title}{family=\librebaskerville,size=\Huge}
\setbeamerfont{subtitle}{family=\librebaskerville,size=\large}
% IMPORTANTE: a fonte 'campus' non queda ben para títulos de papers academicos,ç
% pero se de verdade se desexa empregar, seguir os seguintes pasos
% 1) Instalar o comando otftotfm en linux
% 2) sudo otftotfm -a -e texnansx campus_bold.otf CampusBold
% 3) asegurarse que o ficheiro auxiliar ./EETtemplateFiles/fonts/T1CampusBold.df está no directorio de traballo
% 4) descomentar a liña abaixo e comentar a liña que lle asigna librebaskerville arriba
\input{EETtemplateFiles/fonts/T1CampusBold.df}
\setbeamerfont{title}{family=\fontfamily{CampusBold},size=\Huge}

% DETLARACIÓN DO TEMA A USAR
% 
% ESTES TEMAS TEÑEN CABECEIRAS MOI GRANDES
% \usetheme{Berkeley} %large titlebar w/side dossier
% \usetheme{PaloAlto} 
% \usetheme{Copenhagen} %large titlebar w/2 side index
% \usetheme{Antibes} %large titlebar w/tree
% \usetheme{Singapore} %large titlebar w/balls evanescent
% \usetheme{Berlin} %large titlebar w/balls solid
% \usetheme{Dresden} %same as above with different color boxing
% \usetheme{Rochester} %large tittle-only titlebar
% ESTES TEMAS TEÑEN CABECEIRAS MEDIANAS
% \usetheme{CambridgeUS} %title titlebar w/current section
% \usetheme{Malmoe} %title titlebar w/current section Copenhagen style
% \usetheme{Madrid} %title titlebar w/page counter footer
% ESTES TEMAS TEÑEN CABECEIRAS DELGADAS
% \usetheme{Frankfurt} %small titlebar w/ progress balls
% \usetheme{metropolis} %metal
% ESTES TEMAS NON TEÑEN CABECEIRA DE COR, PERO SI TITULO SOBRE BRANCO
% \usetheme{Boadilla} %sombras e decoracion
\usetheme{Pittsburgh} %rectangulos planos
% ESTES TEMAS TEÑEN INDICES OU INFO NUNHA BARRA LATERAL GRANDE
% \usetheme{Goettingen} %right dossier evanescent
% \usetheme{Marburg} %right dossier fading to black
% \usetheme{Bergen} %notebook

%aspect modifiers
\useinnertheme{circles} %this makes item lists nicer
% \useoutertheme{infolines} %toggle thin info borders


% DECLARACIÓN DA COMBINACIÓN DE CORES A USAR. SE NON SE ESPECIFICA NADA TOMA A DEFINIDA POR DEFECTO
\definecolor{EETblue}{HTML}{0094e0} % a mate dark blue
\usecolortheme[named=EETblue]{structure} % EET UVigo blue
% outros temas de cores de beamer
% \usecolortheme{seagull} %makes title boxes gray color with blackr
% \usecolortheme{spruce} %makes title boxes pastel blue - gray color
% cores internos (items)
% \usecolortheme[named=Red]{structure} 
% \usecolortheme[named=Green]{structure} 
% \usecolortheme[named=OliveGreen]{structure} 
% \usecolortheme[named=PineGreen]{structure} 
% \usecolortheme[named=TealBlue]{structure} 
% \usecolortheme[named=SeaGreen]{structure}
% \usecolortheme[RGB={00,78,135}]{structure} % a dark cobalt blue 
% \usecolortheme[RGB={155,0,20}]{structure} % a slighlty darkened mate red

% MODIFICACIONS DAS CORES PARA A PAXINA DE TITULO SIMILAR Á OFICIAL
\setbeamercolor*{title}{use=structure,fg=structure.bg, bg=structure.fg}  
\setbeamercolor*{subtitle}{use=structure,fg=white}  
\setbeamercolor*{author}{use=structure,fg=structure.fg}
% \setbeamercolor*{institute}{use=structure,fg=structure.fg}
\setbeamercolor*{date}{use=structure,fg=structure.fg}
\setbeamertemplate{frametitle}[default][left]

% MODIFICACIONS DA SIDEBAR E FOOTLINE PARA INCLUIR AS IMAXES CORPORATIVAS.
\setbeamertemplate{footline}[text line]{%
  \parbox{\linewidth}{
    %ESTE TEXTO DA FOOTLINE PODESE MODIFICAR A GUSTO -----------------------------------------
    \insertshorttitle\hfill\insertshortauthor\hfill\insertpagenumber / \inserttotalframenumber
    %----------------------------------------------------------------------------------------
    \hfill
  \includegraphics[width=.15\paperwidth,trim={0 2.5cm 3.5cm .75cm},clip]{EETtemplateFiles/img/Logotipo_ESCOLA.pdf}\vspace*{2pt}}}
\setbeamersize{sidebar width left = .10\paperwidth}
\setbeamertemplate{sidebar canvas left}{}
\setbeamertemplate{sidebar left}{%
  \vspace*{\fill}
  \includegraphics[width=.15\paperwidth,height=.15\paperwidth]{EETtemplateFiles/img/Simbolo_ESCOLA.pdf}\\
  \includegraphics[width=.15\paperwidth,trim={.4cm .5cm .4cm 2.25cm},clip]{EETtemplateFiles/img/Logotipo_ESCOLA.pdf}
  \vspace*{-11pt}%
}
