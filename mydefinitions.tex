% MATH SYMBOLS

\newcommand{\field}[1]{\mathbb{#1}}

\DeclareMathOperator{\atan}{atan}
\DeclareMathOperator{\acos}{acos}
\DeclareMathOperator{\asin}{asin}

% \newcommand{\mb}[1]{\mathbf{#1}}


\newcommand{\A}{\mathbf{A}}
\newcommand{\B}{\mathbf{B}}
\newcommand{\Cb}{\mathbf{C}}
\newcommand{\D}{\mathbf{D}}
\newcommand{\Eb}{\mathbf{E}}
\newcommand{\F}{\mathbf{F}}
\newcommand{\Gb}{\mathbf{G}}
\newcommand{\Hb}{\mathbf{H}}
\newcommand{\I}{\mathbf{I}}
\newcommand{\J}{\mathbf{J}}
\newcommand{\Kb}{\mathbf{K}}
\newcommand{\Lb}{\mathbf{L}}
\newcommand{\M}{\mathbf{M}}
\newcommand{\N}{\mathbf{N}}
\newcommand{\Ob}{\mathbf{O}}
\newcommand{\Pb}{\mathbf{P}}
\newcommand{\Q}{\mathbf{Q}}
\newcommand{\R}{\mathbf{R}}
\newcommand{\Sb}{\mathbf{S}}
\newcommand{\T}{\mathbf{T}}
\newcommand{\U}{\mathbf{U}}
\newcommand{\V}{\mathbf{V}}
\newcommand{\W}{\mathbf{W}}
\newcommand{\X}{\mathbf{X}}
\newcommand{\Y}{\mathbf{Y}}
\newcommand{\Z}{\mathbf{Z}}
\newcommand{\Dl}{\mathbf{\boldsymbol{\Delta}}}
\newcommand{\Sg}{\mathbf{\boldsymbol{\Sigma}}}
\newcommand{\Ld}{\mathbf{\boldsymbol{\Lambda}}}
\newcommand{\Ph}{\mathbf{\boldsymbol{\Phi}}}
\newcommand{\Ps}{\mathbf{\boldsymbol{\Psi}}}
\newcommand{\Up}{\mathbf{\boldsymbol{\Upsilon}}}
\newcommand{\Xib}{\mathbf{\boldsymbol{\Xi}}}
% \newcommand{\D}{\mathbf{D}}
\newcommand{\one}{\mathbf{1}}
\newcommand{\zero}{\mathbf{0}}

\newcommand{\ab}{\mathbf{a}}
\newcommand{\bb}{\mathbf{b}}
\newcommand{\cc}{\mathbf{c}}
\newcommand{\dd}{\mathbf{d}}
\newcommand{\e}{\mathbf{e}}
\newcommand{\f}{\mathbf{f}}
\newcommand{\g}{\mathbf{g}}
\newcommand{\h}{\mathbf{h}}
\newcommand{\ib}{\mathbf{i}}
\newcommand{\jb}{\mathbf{j}}
\newcommand{\kb}{\mathbf{k}}
\newcommand{\lb}{\mathbf{\ell}}
\newcommand{\m}{\mathbf{m}}
\newcommand{\n}{\mathbf{n}}
\newcommand{\ob}{\mathbf{o}}
\newcommand{\pp}{\mathbf{p}}
\newcommand{\q}{\mathbf{q}}
\newcommand{\rr}{\mathbf{r}}
\newcommand{\s}{\mathbf{s}}
\newcommand{\uu}{\mathbf{u}}
\newcommand{\vv}{\mathbf{v}}
\newcommand{\w}{\mathbf{w}}
\newcommand{\x}{\mathbf{x}}
\newcommand{\y}{\mathbf{y}}
\newcommand{\z}{\mathbf{z}}
\newcommand{\al}{\mathbf{\boldsymbol{\alpha}}}
\newcommand{\vmu}{\mathbf{\boldsymbol{\mu}}}
\newcommand{\vlambda}{\mathbf{\boldsymbol{\lambda}}}
\newcommand{\vphi}{\mathbf{\boldsymbol{\phi}}}
\newcommand{\vpsi}{\mathbf{\boldsymbol{\psi}}}
\newcommand{\vrho}{\mathbf{\boldsymbol{\rho}}}
\newcommand{\vups}{\mathbf{\boldsymbol{\upsilon}}}
\newcommand{\vxi}{\mathbf{\boldsymbol{\xi}}}

\newcommand{\rank}{\textnormal{rank}}
% \newcommand{\trace}{\textnormal{trace}}
\newcommand{\exptr}{\textnormal{exptr}}
\newcommand{\tr}{\textnormal{tr}}
% \newcommand{\vstack}{\textnormal{vec}}
% \newcommand{\diag}{\textnormal{diag}}
\newcommand{\vstack}[1]{\Xib_{#1}}
\newcommand{\diag}[1]{\Ld_{#1}}
\newcommand{\tnsr}[2]{\underline{\mathsf{#1}}_{#2}}
\newcommand{\tmult}[1]{\underset{#1}{\times}}

\DeclareMathOperator{\Prob}{Prob}
%  |x>
\newcommand{\ket}[1]{\left\vert#1\right\rangle}
%  <x|
\newcommand{\bra}[1]{\left\langle#1\right\vert}
%  <x|y>
\newcommand{\braket}[2]{\left< #1 \vphantom{#2}\,
                        \right\vert\left.\!\vphantom{#1} #2 \right>}
%  <x|a|y>
\newcommand{\sandwich}[3]{\left< #1 \vphantom{#2 #3} \right|
                          #2 \min\left(\vphantom{#1 #2} #3 \right>}

\newcommand{\pd}[2]{\frac{\partial #1}{\partial #2}}
%  d/dt
\newcommand{\ddt}{\frac{d}{dt}}
%  D/Dx
\newcommand{\pdd}[1]{\frac{\partial}{\partial#1}}
%  |x|
\newcommand{\abs}[1]{\left\vert#1\right\vert}
%  k_{x}
\newcommand{\kv}[1]{\mathbf{k}_{#1}}
%  \textnormal{E}_{domain of integration}{variable}
\newcommand{\Ex}[2]{{\mathbb{E}_{#1}\left[#2\right]}}
\newcommand{\CEx}[3]{{\mathbb{E}_{#1}\left[#2|#3\right]}}
\newcommand{\CInf}[3]{{\textnormal{I}\left(#1;#2|#3\right)}}
\newcommand{\Inf}[2]{{\textnormal{I}\left(#1;#2\right)}}
\newcommand{\CEnt}[2]{{\textnormal{H}\left(#1|#2\right)}}
\newcommand{\Ent}[1]{{\textnormal{H}\left(#1\right)}}
\newcommand{\dCEnt}[2]{{\textnormal{h}\left(#1|#2\right)}}
\newcommand{\dEnt}[1]{{\textnormal{h}\left(#1\right)}}

\newcommand{\cmark}{\ding{51}}%
\newcommand{\xmark}{\ding{55}}%
\newcommand{\itempro}{\item[\textcolor{KYJade}{\Large \cmark}]}
\newcommand{\itemcontra}{\item[\textcolor{ARust}{\Large \xmark}]}
\newcommand\Tau{\mathcal{T}}
%Figure and format fixes


\renewcommand{\figurename}{Fig.}
\newcommand{\PESrule}{\noindent\rule{.57\columnwidth}{0.1mm}}

%theroem environments
% If using amsthm package, we need to delete these theorems before giving them our own definition. does not work for theorem
% \let\theorem\relax
\let\definition\relax
\let\lemma\relax
\let\corollary\relax
\let\example\relax
%
% \newtheorem{theorem}{Theorem}
\newtheorem{definition}{Definition}
\newtheorem{lemma}{Lemma}
\newtheorem{corollary}{Corollary}
\newtheorem{conjecture}{Conjecture}
\theoremstyle{plain}
\newtheorem{remark}{Remark}
\newtheorem{proposition}{Proposition}
\newtheorem{example}{Example}
\newtheorem{homework}{Homework}

%Colors
   \definecolor{blueH3}{rgb}{0,.5,1}
   \definecolor{blueH2}{rgb}{0,0.25,0.75}
   \definecolor{blueH1}{rgb}{0,0,0.5}   
   \definecolor{grayOldText}{rgb}{.5,.5,.5}
   \definecolor{VCobalt}{HTML}{005682}
   \definecolor{TZTeal}{HTML}{008080}
   \definecolor{TZTealfaded}{HTML}{F0FFFF}
   \definecolor{KYJade}{HTML}{008151}
   \definecolor{ARust}{HTML}{a10000}
   \definecolor{FFucsia}{HTML}{7000c3}   
   \definecolor{TAMustard}{HTML}{a1a100}
   \definecolor{Tangerine}{HTML}{d45500}
   
   
% Tikz 
% signal block diagram components
\tikzset{
    block/.style = {draw, rectangle, 
        minimum height=1cm, 
        minimum width=1.2cm, align=center},
    input/.style = {coordinate,node distance=1cm},
    output/.style = {coordinate,node distance=2cm},
    arrow/.style={draw, -latex,node distance=1.5cm},
    pinstyle/.style = {pin edge={latex-, black,node distance=1.5cm}},
    sum/.style = {draw, circle, node distance=1cm}
}
\tikzstyle{pinstyle} = [pin edge={to-,thin,black}]
\def\antenna{%
    -- +(0mm,4.0mm) -- +(2.625mm,7.5mm) -- +(-2.625mm,7.5mm) -- +(0mm,4.0mm) -- +(0mm,0mm)
}
% Overlay highlights on top of the page
\newcommand{\markOverlay}[1]{\tikz[overlay,remember picture] \node (#1) {};}
\newcommand{\drawOverlayBox}[4][]{%
    \tikz[overlay,remember picture]{%
        \coordinate (TopLeft)     at ($(#2)+(-0.4em,1.6em)$);
        \coordinate (BottomRight) at ($(#3)+(0.4em,-1.0em)$);
        %
        \path (TopLeft); \pgfgetlastxy{\XCoord}{\IgnoreCoord};
        \path (BottomRight); \pgfgetlastxy{\IgnoreCoord}{\YCoord};
        \coordinate (LabelPoint) at ($(\XCoord,\YCoord)!0.5!(BottomRight)$);
        %
        \draw [red,#1] (TopLeft) rectangle (BottomRight);
        \node [below, #1, fill=none, fill opacity=1] at (LabelPoint) {#4};
    }
}
\newcommand{\drawOverlayLine}[4][]{%
    \tikz[overlay,remember picture]{%
        \draw [red,#1] ($(#2)$) -- node{#4} ($(#3)$);
    }
}
\newcommand{\drawOverlayCircle}[4][]{%
    \tikz[overlay,remember picture]{%
        \draw [red,#1] ($(#2)$) circle (#3) node{#4};
    }
}
   
   %%%%%%%%%%%%%%%%%%%%%%%%%%%%%%%%%%%%%%%%%%%%%%%%%%%%%%%%%%%%%%%%%
%% The following definitions are to extend the LaTeX algorithmic 
%% package with SWITCH statements and one-line structures.
%% The extension is by 
%%   Prof. Farn Wang 
%%   Dept. of Electrical Engineering, 
%%   National Taiwan University. 
%% 
\newcommand{\SWITCH}[1]{\STATE \textbf{switch} (#1)}
\newcommand{\ENDSWITCH}{\STATE \textbf{end switch}}
\newcommand{\CASE}[1]{\STATE \textbf{case} #1\textbf{:} \begin{ALC@g}}
\newcommand{\ENDCASE}{\end{ALC@g}}
\newcommand{\CASELINE}[1]{\STATE \textbf{case} #1\textbf{:} }
\newcommand{\DEFAULT}{\STATE \textbf{default:} \begin{ALC@g}}
\newcommand{\ENDDEFAULT}{\end{ALC@g}}
\newcommand{\DEFAULTLINE}[1]{\STATE \textbf{default:} }
%% 
%% End of the LaTeX algorithmic package extension.

\newcounter{MYtempeqncnt}


%%%%%%%%%%%%%%%%%%%%%%%%%%%%%%%%%%%%%%%
% Commands to recall text later
%%%%%%%%%%%%%%%%%%%%%%%%%%%%%%%%%%%%%%%
\makeatletter
\newcommand\remembertext[2]{% #1 is a key, #2 is the text
  \immediate\write\@auxout{\unexpanded{\global\long\@namedef{mytext@#1}{#2}}}%
  #2%
}
%
\newcommand\recalltext[1]{%
  \ifcsname mytext@#1\endcsname
    \@nameuse{mytext@#1}%
  \else
    ``??''
  \fi
}

%%%%%%%%%%%%%%%%%%%%%%%%%%%%%%%%%%%%%%%%%%%%%%%%%%%%%%%%%%%%%%%%%%%%%%%%%%%%%%%%%%
%%% Paolo Casari: macros for automating section titling and comment formatting %%%
%%%%%%%%%%%%%%%%%%%%%%%%%%%%%%%%%%%%%%%%%%%%%%%%%%%%%%%%%%%%%%%%%%%%%%%%%%%%%%%%%%
\newcounter{myequationcnt}

\newcounter{rcnt}
\newcounter{ccnt}

\newcommand{\newreviewernopagebreak}[1]{\vspace{5em} \setcounter{ccnt}{0}\section*{\normalsize Comments of #1}\vspace{4mm}}

\newcommand{\ThisIsTheEditorNoPageBreak}{\setcounter{ccnt}{0}\section*{\Large Comments of the Editor}\vspace{3mm}}
\newcommand{\ThisIsTheEditor}{\clearpage \ThisIsTheEditorNoPageBreak}
\newcommand{\ThisIsANewReviewerNoPageBreak}[1]{\vspace{5em} \refstepcounter{rcnt}\label{r#1}\setcounter{ccnt}{0}\section*{\Large Comments of Reviewer \arabic{rcnt}}\vspace{3mm}}
\newcommand{\ThisIsANewReviewer}[1]{\clearpage\vspace{-5em} \ThisIsANewReviewerNoPageBreak{#1}}

\newcommand{\edcomment}[1]{
\begin{tcbremark}
\color{VCobalt}
    \refstepcounter{ccnt}\label{e\arabic{ccnt}}\noindent\textbf{\boldmath\emph{Comment E.\arabic{ccnt}:}} #1\vspace{0.2cm}
\end{tcbremark}
}
\newcommand{\refedcomment}[1]{E.\ref{e#1}}

\newcommand{\revcomment}[1]{
\begin{tcbremark}
\color{VCobalt}
\refstepcounter{ccnt}\label{r\arabic{rcnt}c\arabic{ccnt}}\noindent\textbf{\boldmath\emph{Comment \arabic{rcnt}.\arabic{ccnt}:}} #1\vspace{0.2cm}
\end{tcbremark}
}
\newcommand{\refrevcomment}[2]{\ref{r#1}.\ref{r#1c#2}}

% \newcommand{\ouranswer}[1]{\noindent\emph{Answer:} #1\vspace{0.6cm}}
% \newcommand{\citepap}[1]{\vspace{0.33cm}\begin{minipage}{0.05\textwidth} $\phantom{A}$  \end{minipage}\begin{minipage}{0.85\textwidth}\renewcommand{\baselinestretch}{1.15}\small \emph{#1} \end{minipage}\vspace{0.3cm}}

\newlength{\ansspace}
\addtolength{\ansspace}{0.6cm}
\newcommand{\ansbreak}{\vspace{\ansspace}}

\newlength{\stdleftskip}
\addtolength{\stdleftskip}{\leftskip}
\newlength{\stdrightskip}
\addtolength{\stdrightskip}{\rightskip}
\newlength{\citeskip}
\addtolength{\citeskip}{2em}
\newcommand{\oldbaselinestretch}{1.5}

\newcommand{\setcitepapskip}{%
    \leftskip\citeskip %
    \rightskip\citeskip %
    \renewcommand{\baselinestretch}{1.15}\small%
    \vspace{0.6em}%
    \noindent%
}

\newcommand{\resetLRmargins}{%
    \leftskip\stdleftskip %
    \rightskip\stdrightskip %
    \renewcommand{\baselinestretch}{\oldbaselinestretch}\normalsize %
    \vspace{0.6em}
}

\newcommand{\emans}{\emph{Answer:\ }}
